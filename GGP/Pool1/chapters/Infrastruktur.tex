\section{Infrastruktur}

\subsection{Aktuelle/Neue Projekte}

Im Bereich Infrastruktur ist der öffentliche Nahverkehr ein wichtiger Punkt. Durch den Ausbau der U-Bahn und Straßenbahn Linien ist es immer einfacher möglich schnell und ohne Auto durch die Stadt zu kommen. Im Kern der Stadt werden gerade die Linie U2 ausgebaut und die Linie U5 neu gebaut. Beide sind eine wichtige Ergänzung des aktuellen Liniennetzes. Die Erweiterung der U2 bindet Randgebiete der Stadt an das Liniennetz an und die U5, welche auch bald autonom, also Fahrerlos fährt bietet eine schnelle Verbindung zwischen dem Karlsplatz und Hernals, jedoch erst ab 2035, in der näheren Zukunft fährt die Linie nur bis zum Frankhplatz.

Auch im Bereich der Straßenbahnen entwickelt sich aktuell  viel. Es werden 3 neue Linien gebaut, die Linien 12, 27 und 72 und weitere 3 verlängert. Damit wird die Seestadt an das Verkehrsnetz der öffentlichen Verkehrsmittel angeschlossen und in der Leopoldstadt, in Floridsdorf und in Meidling das Angebot erweitert. So können Personen, vorallem in Randbezirken, schnell und komfortabel in die Stadt kommen. Zuletzt wurden die Linien O vom Praterstern bis zur Bruno-Marek-Allee durch den zweiten Wiener Gemeindebezirg und D vom Hauptbahnhof bis zur Absberggasse verlängert.

\subsection{Zukunft}

\subsubsection{Öffentliche Verkehrsmittel}

Vorallem bei den öffentlichen Verkehrsmitteln wird in den nächsten Jahren viel geschehen. Es könnten nach den ersten autonom fahrenden Ubahnen auch Straßenbahnen und sogar Busse dazu kommen. Dadurch könnte die Fehleranfälligkeit von Menschen umgangen werden. Weiters wird ein Rückgang des privaten Verkehrs geschehen und mehr auf öffentlichen Verkehr gesetzt werden, was, vorallem bei Straßenbahnen und Bussen, also bei allen oberirdisch Verkehrenden Transportmitteln, die Pünktlichkeit erhöht, da diese nicht mehr so oft in Staus stehen. Für den privaten Verkehr erweitert sich der Markt für Ride-sharing immer weiter. Immer weniger Menschen werden eigene PKWs haben, aber immer mehr Autos, welche man schnell ausborgen kann stehen an den Straßenrändern. In der weiteren Zukunft werden solche Ridesharing Angebote nur noch aus Selbstfahrenden Fahrzeugen bestehen, diese dann gemietet werden können.

\subsubsection{Strom und Wasser Versorgung}

Weiters müssen in Wien die Strom und Wasserversorgung überarbeitet werden. Durch Überlastungen und Störungen fällt immer öfter die Stromversorgung in der Stadt aus. Diese kann schnellst möglich durch etwa redundante Leitungen gesichert werden und so für eine Zukunft mit immer höherem Stromverbrauch bereit gemacht werden. Die Wasserversorgung ist deutlich wichtiger als die Stromversorgung, da diese Lebensnotwendig ist. Durch menschliche Missgeschicke oder eine Wasserunterversorgung kann die städtische Bevölkerung in Gefahr geraten, da zu wenig oder gar kein Wasser vorhanden ist. Die 1. Wiener Hochquellleitung wurde 1873 erbaut, wird zwar gewartet, muss aber in der Zukunft erneuert werden, damit eine zuverlässige Wasserversorgung besteht. Es müssen auch in Zukunft durch die Erweiterung mehr Hauptwasserversorgungen gebaut werden, wie die 3. Hauptversorgungsleitung Wien Nord, welche 2026 in Döbling eröffnet wird.
Eine Herausforderung wird es neue Wohngebiete am Rand der Stadt zuverlässig mit Wasser, Strom und öffentlichen Anbindungen zu versorgen. Da immer mehr Wohngebiete wie die Seestadt entstehen müssen dorthin auch Leitungen und Rohre mit hoher Kapazität verlegt werden. Durch das wird die Auslastung auf das aktuelle Netz erhöht und Erweiterungen sind möglicherweise nötig.

\subsubsection{medizinische Versorgung}

Es muss auch dringend die medizinische Versorgung angepasst werden und neue Krankenhäuser, besonders mit einer Unfall Ambulanz errichtet werden, da das Lorenz-Böhler UKH geschlossen wurde. Mit neuen Wohnräumen müssen auch neue Krankenhäuser, Schulen und Parks errichtet werden. Dadurch werden zwar Arbeitsplätze geschaffen, jedoch sind diese mit einer relativ Aufwändigen bis sehr Aufwändigen Ausbildung verbunden.

\subsubsection{Einzelverkehr}

Die Stadt Wien ist und soll beziehungsweise wird auch weiter den Radverkehr fördern und damit, vorallem in den innerstädtischen Bezirken autofreie Straßen einrichten und Breite Radwege durch die Stadt bauen. Das fördert den individuellen Personenverkehr und ist dazu noch Klimaneutral.
Durch die Klimaerwärmung wird es, vorallem in Städten immer wärmer. Durch Bodenversiegelung und wenige Pflanzen in der Nähe steigt die Temperatur immer weiter. Durch Einrichtung von Parks und Bäumen am Rand von Straßen kann die Luft deutlich gekühlt und für den Klimaschutz etwas getan werden. Auch entsteht dadurch natürlicher Schatten. Das sollte vorallem in innerstädtischen Bezirken durchgeführt werden, da dort kein Zugang zu kühlen Orten besteht.

\subsection{Probleme}

Bei diesen Entwicklungen können Probleme entstehen, wie etwa Verzögerungen der Bauvorhaben, politische Differenzen zwischen Parteien oder auch Platz- und Geldprobleme. Weiters kann der Flächenwidmungsplan zu Problemen auf manchen Flächen führen, da zum Beispiel Bausperren vorliegen.

\subsection{Zusammenfassung}

Die Stadt Wien arbeitet in vielen Bereichen in eine Zukunftssichere Richtung, jedoch kann noch einiges ausgebaut werden. Etwa könnte mehr Platz für den Fahrradverkehr und weniger für den Autoverkehr verwendet werden. Auch die öffentlichen Verkehrsmittel werden ausgebaut, jedoch könnten auch andere Strecken noch verlängert beziehungsweise saniert werden.
