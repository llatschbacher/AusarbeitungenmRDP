\section{Stadtentwicklung}

\subsection{Aktuelle Situation}

In Wien gibt es aktuell viele Bauprojekte im Rahmen der Wohnraumschaffung. Es werden viele Neubauten oder sogar ganze Siedlungen, wie die Seestadt gebaut um Wohnraum für die immer mehr werdende Bevölkerung zu schaffen. In den meisten Wiener Gemeindebezirken gibt es Bauprojekte die bevorstehen. Auch neue Gemeindebauten, vorallem in Favoriten werden errichtet, mehr leistbare Wohnungen in Wien entstehen. Diese günstigen Wohnungen sind besonders wichtig für Staatsbürger mit einem geringerem Einkommen, damit diese nicht obdachlos werden und auf der Straße leben müssen.  Weiters sind auch Projekte wie eine Nachbegrünung in der Seestadt gerade in Arbeit. 

\subsection{Zukunft}

Eine Entwicklung der Stadt wird aus sicht der Flächenwidmung und der Bebaubarkeit der Gebiete im Norden, Süden und Osten stattfinden, da im Westen die Landschaft zu hügelig für große Siedlungen ist. Durch Zuwanderung von Flüchtlingen, wird es notwendig sein mehr Wohnraum zu schaffen und dieser muss zukunftsorientiert gestaltet werden, also offen und mit viel begrünung. Weiters ist der Ausbau von der Infrastruktur in diesen Gebieten notwendig, wie im Kapitel Infrastruktur beschrieben ist nötig. Durch die erhöhte Zuwanderung aus Kriegsländern steigt auch die Geburtenrate wieder und die Sterberate wird dadurch etwas mehr kompensiert. Durch eine höhere Anfrage als das Angebot steigen Mieten in der Zukunft, da zu viele Menschen und zu wenig Wohnungen vorhanden sind. Momentan sind zwar die Mietpreise stagniert, aber in der Zukunft steigen die Mietpreise weiter. Auch die soziale Durchmischung wird durch die Mietpreise beeinflusst, die wohlhabenderen Menschen in manchen Bezirken wie dem 1. oder 19. Bezirk und Personen beziehungsweise Familien mit weniger Geld werden eher in günstigeren Bezirken leben, was einen sozialen Spalt weiter aufkommen lässt. Einkommensschwache Bewohnerinnen und Bewohner werden, wenn sie keinen passenden Wohnraum finden verdrängt und müssen sich in einem günstigerem Gebiet einen Wohnraum suchen. Genau deshalb sind günstige Wohnangebote wichtig zu schaffen wie die Gemeindebauten in Wien.
Es gibt auch in Wien viele leerstehende Wohnräume, dies können entweder für eine andere Nutzung freigegeben werden oder auch für soziale Projekte freigeben. Das schont die städtischen Ressourcen und bietet zugleich Hilfe für Bedürftige.

\subsection{Zusammenfassung}

Bis jetzt genutze Wohnräume müssen umfunktioniert werden und mehr günstige Wohnungen am Rand der Stadt gebaut werden. Diese sollten offen und mit viel Begrünung eingerichtet werden. Auch Sozialprojekte für Zuwandernde sind ein wichtiger Punkt. Weiters müssen mehr Gemeindebauten entstehen, damit für alle ein bezahlbarer Wohnraum verfügbar ist.