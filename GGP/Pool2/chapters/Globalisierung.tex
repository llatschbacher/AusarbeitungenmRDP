\section{Globalisierung}

\subsection{Auswirkungen auf die Wirtschaft in Österreich}

Durch die Globalisierung wurde Österreich der Zugang zu internationalen Märkten erleichtert und Exportmöglichkeiten für österreichische Unternehmen wurden erweitert. Weiters wurde der Wettbewerb auf dem Weltmarkt verschärft, daher ist auf dem internationalen und regionalen Markt das Angebot größer. Regionale Unternehmen stehen vor einem größeren Wettkampf mit ausländischen Unternehmen. Dadurch entsteht Innovationsdruck und die Effizienz wird gesteigert.
Durch Investitionen aus dem Ausland und im Ausland entstehen neue Arbeitsplätze und zur Stärkung der Wirtschaft wir beigetragen. Auch auf dem österreichischen Arbeitsmarkt sieht man die Auswirkungen der Globalisierung. Zum einen gibt es eine erhöhte Nachfrage nach hochqualifizierten Arbeitskräften in exportorientierten Branchen, jedoch sind in anderen Sektoren Arbeitsplätze weggefallen, da durch Automatisierung und Verlagerung der Produktionen die Arbeitsplätze für Menschen wegfallen.
Österreich profitiert auch durch die technologischen Fortschritte und Innovationen von anderen Ländern durch die Globalisierung. Insbesondere in den Branchen IT und Forschung haben diese ausländischen Entwicklungen zum Wirtschaftswachstum beigetragen.

\subsection{Wirtschaftsräume in Österreich}

\subsubsection{Ballungszentren}

\begin{itemize}
	\item \textbf{Wien:} Wien ist die Hauptstadt von Österreich und ist damit nicht nur das politische Zentrum des Landes, sondern auch das größte Wirtschaftszentrum. Es gibt in der Hauptstadt viele internationale Organisationen, Unternehmen aus verschiedenen Branchen sowie ein breites Angebot an Dienstleistungen. Die Wirtschaft Wiens ist äußerst vielfältig und umfasst Branchen wie Finanz- und Versicherungsdienstleistungen, Informations- und Kommunikationstechnologie, Tourismus sowie kreative Industrien wie Design und Werbeagenturen. Außerdem ist Wien ein bedeutender Standort für Forschung und Entwicklung.. Die Stadt beherbergt mehrere renommierte Universitäten und Forschungseinrichtungen.
	\item \textbf{Linz:} Linz, die drittgrößte Stadt Österreichs, ist ein Industrie- und Technologiestandort, welcher eine wichtige Rolle in der österreichischen Wirtschaft spielt. Die Stadt ist bekannt für ihre Stahl- und Chemieindustrie sowie für ihre Forschungs- und Entwicklungseinrichtungen. Neben der traditionellen Industrie wächst in Linz immer weiter die Branchen der Informationstechnologie und Elektronik.
	\item \textbf{Graz:} Die Stadt Graz ist ein bedeutendes Wirtschaftszentrum in der Steiermark und gilt als Standort für innovative Unternehmen, insbesondere im Bereich der Elektronik, Maschinenbau und Automobilindustrie mit dem Unternehmen Magna Steyr. Auch ist Graz ein führender Standort für Forschung und Entwicklung, da etwa die Technische Universität Graz sowie mehrere Forschungsinstitute, welche sich auf die Bereiche der Elektronik, Materialwissenschaften und Automatisierungstechnik spezialisiert haben. Diese Einrichtungen fördern die Zusammenarbeit zwischen Wissenschaft und Wirtschaft und unterstützen die Innovationskraft der Region.
\end{itemize}

\subsubsection{Industrieregionen}

\begin{itemize}
	\item \textbf{Oberösterreich:} Oberösterreich ist bekannt für die vielfältige Industrie, darunter Metallverarbeitung, Maschinenbau, Elektronik, Chemie und ein weltweit relevantes Mobilitätsunternehmen KTM. Die wichtigsten Industriestandorte in dieser Region sind Linz, Wels und Steyr. Diese Regionen sind durch eine hohe Dichte an Industrieunternehmen gekennzeichnet und spielen eine entscheidende Rolle in der österreichischen Wirtschaft.
	\item \textbf{Steiermark:} Die Steiermark ist für die Herstellung von Maschinen und Anlagen bekannt. Die Region verfügt auch über eine bedeutende Agrar- und Lebensmittelindustrie. In der Maschinen Industrie werden in der Steiermark viele Maschinen für die Landwirtschaft und für die Luft- und Raumfahrttechnik hergestellt.
	\item \textbf{Vorarlberg:} Das westlichste Bundesland Österreichs, Vorarlberg, ist bekannt für ihre Textil- und Bekleidungsindustrie, sowie für ihre Holzverarbeitung und den Maschinenbau.
\end{itemize}