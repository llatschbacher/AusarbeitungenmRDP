\section{Grundlagen von Pentesting}

Penetrationtesting oder auch Pentesting ist dafür um Sicherheitslücken in IT-Systemen aufzufinden. Nicht nur Sicherheitslücken sondern auch die Schwierigkeit in ein System einzudringen wird durch einen Pentest bestimmt. Dadurch soll die Sicherheit erhöht werden und vor zukünftigen Angriffen geschützt werden. Sie werden von sogenannten ''ethisch Hackern'' durchgeführt, welche mit einer Genehmigung der Eigentümer Systeme angreifen um Securityissues aufzuspüren, damit diese danach behoben werden können. Die Angriffe finden mit den selben Methoden wie bei Angriffen von nicht ethischen Hackern statt. Weiters gibt es verschiedene Arten von Pentests, welche in einem der folgenden Kapitel erklärt werden.

\subsection{Tools für Pentests}

\subsubsection{Vulnerability Scans}

In der folgenden Auflistung befinden sich die besten Tools für Vulnerability Scans

\begin{itemize}
	\item Acunetix ist ein Web-Vulnerability-Scanner, der über fortschrittliche Crawling-Technologie verfügt, um Schwachstellen zu finden und jeden Typ von Webseite zu durchsuchen – selbst solche, die passwortgeschützt sind.
	\item Wireshark ist ein kostenloses und Open-Source-Paketanalysetool. Es wird für Netzwerkfehlerbehebung, Analyse, Software- und Kommunikationsprotokollentwicklung sowie für Bildungszwecke verwendet. Ursprünglich hieß das Projekt Ethereal, wurde jedoch im Mai 2006 aufgrund von Markenproblemen in Wireshark umbenannt.
	\item Das Metasploit-Projekt ist ein Computersicherheitsprojekt, das Informationen über Sicherheitslücken bereitstellt und bei Penetrationstests und der Entwicklung von IDS-Signaturen hilft. Es gehört der Sicherheitsfirma Rapid7 mit Sitz in Boston, Massachusetts.
	\item OpenVAS ist ein Open-Source-Schwachstellen-Scanner, der von Greenbone Networks gepflegt wird. Der Scanner verfügt auch über einen regelmäßig aktualisierten Community-Feed, der über 50.000 Schwachstellentests enthält.
	\item John the Ripper ist ein kostenloses Passwortcracking-Software-Tool. Ursprünglich für das Unix-Betriebssystem entwickelt, kann es auf fünfzehn verschiedenen Plattformen ausgeführt werden.
	\item sqlmap ist ein Software-Dienstprogramm zur automatischen Entdeckung von SQL-Injektions-Sicherheitslücken in Webanwendungen.
	\item Burp Suite ist ein Web-Schwachstellen-Scanner, der häufig aktualisiert wird und sich mit Bug-Tracking-Systemen wie Jira integriert, um einfache Ticketgenerierung zu ermöglichen.
	\item Aircrack-ng ist eine Netzwerk-Software-Suite, die aus einem Detektor, einem Packetsniffer, einem WEP- und WPA/WPA2-PSK-Knacker sowie einem Analysetool für 802.11 Wireless LANs besteht. Es funktioniert mit jedem drahtlosen Netzwerk-Interface-Controller, dessen Treiber den Raw-Monitoring-Modus unterstützt und kann 802.11a, 802.11b und 802.11g Traffic abhören.
	\item Nessus ist einer der beliebtesten Schwachstellen-Scanner mit über zwei Millionen Downloads weltweit. Darüber hinaus bietet Nessus umfassende Abdeckung und scannt über 59.000 CVEs.
\end{itemize}

\subsubsection{Port Scanning}

\begin{itemize}
	\item Nmap ist ein gratis open source security scanner, der auch von Organisationen für Netzwerkentdeckung, Bestandsaufnahme, Verwaltung von Service-Upgrade-Zeitplänen und Überwachung der Verfügbarkeit von Hosts oder Diensten verwendet wird. Die Features von Nmap sind active port scanning, host discovery, OS detection und application version detection. Active Port scanning erlaubt es in einem Netzwerk beziehungsweise bei bestimmten Hosts nach offenen Ports zu suchen. Host discovery kann Hosts die auf network requests reagieren identifizieren. OS detection kann das Betriebsystem und die Version eines Hosts herausfinden. Dabei können auch Details über das Netzwerk herausgefunden werden. Die Application Version detection kann herausfinden welche Apps laufen und welche Version diese haben.
	\item Unicornscan ist für die Features des asynchronen TCP und UDP scanning mit den ungewöhnlichen Scanpatterns bekannt. Diese bieten alternative Wege um Details über remote Betriebssysteme herauszufinden. Unicornscans Features sind asynchrones, stateless TCP scanning, asynchrones UDP scanning und ein IP port Scanner. Weiters hat die Software wie zuvor erwähnt das Feature aus der ferne das Betriebssystem und die Version dieses zu bestimmen.
	\item Angry IP Scanner ist ein kostenloses, plattformübergreifendes Netzwerk-Scan-tool, das für seine schnelle Scangeschwindigkeit dank seines Ansatzes mit mehreren Threads bekannt ist, der jeden Scan separat durchführt. Es benötigt keinen Installationsvorgang und kann einfach heruntergeladen und ausgeführt werden. Mit Angry IP Scanner können offene Ports in jedem entfernten Netzwerk gescannt werden, und es kann auch Webserver- und NetBIOS-Informationen erkennen. Die Ergebnisse des Scans können in TXT-, XML- oder CSV-Dateien exportiert werden, und es ermöglicht eine einfache Plugin-Integration mit der Java-Sprache.
	\item Netcat ist eines der ältesten Nezwerk-Scan-Tools, die letzte offizielle Version ist aus dem Jahr 2004. Jedoch gibt es mehrere Varianten die auf modernen Systemen funktionieren. Netcat kann TCP und UDP Ports scannen, hat einen eingebauten Port-Scanner und kann über die Commandline benutzt werdne. Weiters sind verschiedene Varianten für Windows, Linux und macOS verfügbar.
	\item Zenmap ist das offizielle GUI für Nmap für Personen, die nicht die Commandline verwenden möchten. Die Scanergebnisse werden in einer Datenbankgespeichert, die danach durchsucht werden kann. Neue Scanergebnisse können mit früheren verglichen werden. Weiters können Port-Scan-Profile für häufig verwendete Porterkennungsoptionen gespeichert werden. Die restlichen Features sind gleich wie bei Nmap, da Zenmap nur ein GUI für Nmap ist.
\end{itemize}

\subsection{Arten von Pentests}

Die verschiedenen Arten von Pentests unterscheiden sich bei den Kosten, der Gründlichkeit und der Menge der vorgegebenen Informationen.

\subsubsection{Black Box}

Bei einem Black Box Penetrationstest bekommt der Pentester im Vorfeld keine Informationen über das zu testende Objekt. Bei dieser Art von Pentest müssen alle Informationen von der Testerin oder dem Tester herausgefunden werden. Diese Art kommt einem wirklichen Angriff am nächsten, ist aber dafür auch am aufwendigsten und teuersten.

\subsubsection{Gray Box}

Der Grey Box Pentest ist eine Kombination aus White Box und Black Box Test. Es werden der testenden Person grundlegende Informationen wie IP-Adressen, Domains und Benutzeraccounts gegeben. Das beschleunigt und vergünstigt den Test, da die Pentesterin oder der Pentester nicht nach diesen Informationen suchen muss. In der Praxis ist das die gängigste Form des Penetrationtests, da diese Art die Balance zwischen Effizienz, Gründlichkeit und Kosten hält.

\subsubsection{White Box}

Bei einem White Box Test werden der Testerin oder dem Tester Informationen über das Prüfobjekt wie Code, Domains, IP-Adressen, Benutzeraccounts oder auch Angaben über die Architektur des zu testenden Objekts bereitgestellt. Weiters ist diese Art des Tests auch sehr gründlich, daher auch sehr teuer. 

\subsubsection{Red Team Assessment}

Bei einem Red Team Assessment Test werden alle verfügbaren Methoden und Techniken verwendet um ein System anzugreifen, daher ist diese Methode ein sehr ressourcenintensives Vorgehen, deshalb kommt es selten zum Einsatz.

\subsubsection{Social-Engineering-Pentest}

Bei dieser Art von Pentest werden nicht nur die herkömmlichen Methoden eines Penetrationstests verwendet, sondern es werden auch durch Social Engineering über die Mitarbeiter Informationen erlangt. Diese Tests sind darauf ausgelegt Schwachstellen in Unternehmen zu finden, welche im Nachhinein durch Schulungen von Mitarbeitenden behoben werden um ernsthafte Angriffe zu erschweren.

\subsection{Phasen eines Pentests}

Die meisten Pentests haben die folgenden Schritte. Diese sind für einen gründlichen und effizienten Penetrationstest notwendig.

\subsubsection{Vorbereitung}

Im ersten Schritt wird der Umfang des Tests ermittelt. Der Kunde und der Tester besprechen welche Objekte zu testen sind, welchen Umfang die Tests haben und wo die Grenzen liegen. Das wird mit einem Dokument vereinbart und eine Einverständniserklärung des Auftraggebers über die Tests wird dem Auftragnehmen gegeben. Es ist wichtig, dass der Umfang der Tests festgelegt ist, damit nicht möglicherweise schon bekannte Schwachstellen gefunden werden, die ohnehin schon bekannt sind, anstatt neue Schachstellen zu finden.

\subsubsection{Informationsbeschaffung}

Bei der Informationsbeschaffung sucht die Testerin oder der Tester nach Informationen über de Netzwerke, Domain Namen, Server und weitere Elemente um das System zu verstehen. Es werden so viele Daten wie möglich gesammelt um eine effektive Strategie zu erstellen. 

\subsubsection{Bewertung der Ergebnisse}

In diesem Schritt werden die zuvor gesammelten Informationen bewertet um eine effektive Strategie zu erstellen. Danach wird eine Strategie erstellt und Tools ausgewählt, welche für den Test verwendet werden. 

\subsubsection{Angriffsversuch}

Im 4. Schritt wird versucht Zugriff auf das System zu erhalten, indem alle Schwachstellen ausgenutzt werden, die zuvor gefunden wurden. Die Arten der Angriffe werden, wie in der zuvor erstellten Strategie festgelegt durchgeführt. Dies richtet sich danach, welche Art von Systemen und Schwachstellen ermittelt werden konnten. Nachdem ein erfolgreicher Zugriff erlangt wurde, wird der Zugriff genutzt um Daten zu stehlen, Netzwerk Traffic abzufangen oder Nutzerrechte zu verändern. Diese Daten werden dazu bentuzt um zu evaluieren, wie groß der Schaden ist, der angerichtet werden könnte, wenn diese Schwachstellen ausgenutzt werden.

\subsubsection{Verfassung eines Berichts}

Der letzte Schritt ist einen Bericht über die durchgeführten Tests zu verfassen. Dieser Bericht beinhaltet normalerweise Schwachstellen, welche Hacker nutzen könnten um Zugriff auf das System zu erlangen. Weiters auch Sensible Daten, die gestohlen werden könntne und Zeitangeben, wie lange es bräuchte um unerkannt im System schaden anrichten zu können.

\subsubsection{Abschluss}

Nachdem der Bericht verfasst wurde, wird dieser dem Auftraggeber übergeben. Dieser wird genutzt um  die gefunden Schwachstellen zu beheben und Sicherheitsupgrades zu implementieren. Das sind zum Beispiel Durchsatzbegrenzungen, DDoS-Abwehr oder auch strengere Formularvalidierungen und -bereinigungen.