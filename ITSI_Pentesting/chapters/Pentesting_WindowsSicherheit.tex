\section{Windows Sicherheit}

\subsection{Windows Sicherheitsarchitektur}

\subsubsection{Benutzerkontensteuerung}

Die Benutzerkontensteuerung (User Account Control (UAC)) ist ein seit Windows Vista eingebauter Sicherheitsmechanismus. Diese Funktion wurde eingeführt, weil viele User mit Admin Privilegien arbeiten und das auf gestartete Anwendungen übertragen wurden. Das stellt ein großes Sicherheitsrisiko dar, weil auch Malware so mit Adminrechten agieren kann. Mit der UAC erhalten alle Programme nur Rechte eines normalen Nutzers, egal welche Rechte der aktive User hat. Bei Bedarf können dem Programm mehr Rechte zugeschrieben werden, das ist durch ein Fenster mit einem Button gelöst. Dieser Mechanismus ist mit dem sudo in UNIX Systemen zu vergleichen.

\subsubsection{Integrierte Sicherheitsfunktionen}

Seit der neuesten Version von Windows wird mit dem Prinzip "Vertraue niemandem" vorgegangen. Das heißt, dass keine Apps und Programme vertrauenswürdig sind, solange das nicht nachgewiesen ist. Das Konzept beginnt beim Start von Windows 11. Dabei sorgt das TPM (Trusted Plattform Module) zusammen mit Secure Boot und UEFI dafür, dass das OS nur startet, wenn es nicht manipuliert wurde. TPM ist ein zusätzlicher Chip, der mittels Kryptografieschlüssel überprüft, ob das OS und die Firmware auch die Richtigen sind nicht etwas anderes, was es nur vorgibt OS oder Firmware zu sein.

\subsubsection{Verschlüsselung}

In Windows gibt es zur Verschlüsselung der Daten den sogenannten "BitLocker". Dieser ist auf dem Windows Laufwerk installiert und kann den ganzen PC oder einzelne Laufwerke verschlüsseln. Ein Passwort kann gesetzt werden um die Verschlüsselung aufzuheben und unbefugten Zugriff zu vermeiden. Der BitLocker beruht auf der Verwendung eines Hardwareelements namens TPM. Der Bitlocker erstellt einen Wiederherstellungschlüssel für die Festplatte, sodass beim Einschalten des Computers ein PIN erforderlich ist, um Zugriff zu erhalten. Außerdem gibt einen Wiederherstellungsschlüssel, der verwendet werden kann, wenn das Passwort vergessen wird.

\subsubsection{Authentifizierung und Zugriffskontrolle}

In Windows gibt es mehrere Authentifizierungsmöglichkeiten die als Zugriffskontrolle auf Elemente und Dateien dienen.

Für die Zugriffskontrolle mit einem Passwort gibt es Passwortrichtlinien. Für Passwörter für normale User gibt es keine festgelegten Richtlinien in Windows, jedoch können Systemadministratoren Passwortrichtlinien festlegen. Dabei gibt seitens Microsoft Empfehlungen, welche besagen, dass Passwörter mindestens acht Zeichen lang sein sollen, gängige Passwörter wie 1234 verboten werden sollten und eine Registrierung für eine Multifaktorauthentifizierung erzwungen wird.

Weiters gibt es die Multifaktorauthentifizierung (MFA), welche mehrere Identitätsnachweise wie Tokens, biometrische Daten oder Passwörter erfordert.

Die Zugriffskontrolle kann vom Systemadministrator konfiguriert werden. Es können Usergruppen erstellt werden, um sicherzugehen, dass nur autorisierte User auf bestimmte Ressourcen zugreifen können und dass der Zugriff auf sensible Daten eingeschränkt ist.

\subsubsection{Patch-Management}

Patch-Management in Windows ist der Prozess der Verwaltung und Bereitstellung von Software-Updates, um Sicherheitslücken zu schließen, Fehler zu beheben und die Systemstabilität zu verbessern. Es umfasst die Identifizierung von Schwachstellen, die Planung von Patch-Zyklen, das Testen von Patches, die Bereitstellung auf produktiven Systemen, die Überwachung der Patch-Compliance und gegebenenfalls das Rückgängigmachen von Patches bei Problemen. Ein effektives Patch-Management ist entscheidend, um die Sicherheit und Leistung von Windows-Systemen zu gewährleisten.

\subsubsection{Sicherheitsüberwachung und Protokollierung}

Durch Ereignisprotokolle werden zum Beispiel Anmeldeversuche, Zugriffsfehler oder Änderungen an kritischen Systemeinstellung aufgezeichnet und gespeichert. Um zu garantieren, dass diese Protokolle nicht gelöscht werden oder durch Fehler verschwinden sollten diese auf einem anderen System gespeichert werden, damit im Ernstfall ein Backup davon besteht. Es kann auch festgelegt werden was protokolliert wird. Es können folgende Ereignisse in der Protokollierung aktiviert werden. Das sind Account Logon, Account Management, Detailed Tracking, DS Access, Logon/Logoff, Object Access, Policy Change, Privilege Use, Syste und Global Object Access Auditing.

\subsubsection{Virtualisierung und Isolation}

Virtualisierungsbasierte Sicherheit (VBS) nutzt Hardwarevirtualisierung und den Windows Hypervisor, um eine isolierte virtuelle Umgebung für das Hosten von Sicherheitslösungen zu schaffen und so Schutz vor Schwachstellen im Betriebssystem zu bieten. 
Es gibt 2 verschiedene Isolationsmodi, Prozessisolation und Hyper-V-Isolierung. Bei der Prozessisolation werden pro Container verschiedene Namespaces isoliert, das sind das Dateisystem, die Registrierung, die Netzwerkports, Prozess und Thread-ID-Bereich und der Objekt-Manager-Namespace.
Die Hyper-V-Isolierung bietet eine erhöhte Sicherheit und umfassende Kompatibilität zwischen Host- und Containerversion. Dabei werden mehrere Containerinstanzen gleichzeitig auf dem Host ausgeführt, wobei jeder Container innerhalb eines hochgradig optimierten virtuellen Computer ausgeführt und erhält seinen eigenen Kernel. Durch den virtualisierten Computer kann die Isolation auf Hardwareebene zwischen den einzelnen Containern und dem Host stattfinden.

\subsubsection{Sicherheitsrichtlinien und Compliance}

\subsection{Methoden des Pentestings in Windows}

\subsubsection{Netzwerk-Pentesting}

\subsubsection{Web-Pentesting}

\subsubsection{Angriff auf die Benutzerkontensteuerung (UAC) und privilegierte Eskalation}

\subsubsection{Social Engineering-Angriffe gegen Windows-Benutzer}

\subsection{Praxisbeispiele}
